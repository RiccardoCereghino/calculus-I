% Chapter 6

\chapter{Limiti} % Chapter title

\label{ch:limiti} % For referencing the chapter elsewhere, use \autoref{ch:name}

%----------------------------------------------------------------------------------------

\section{Punto di accumulazione}
Dato un insieme $A \subseteq \R$ e:
\begin{enumerate}
\item un punto $x_0 \in \R$ è detto punto di accumulazione per $A$ se per ogni $\delta > 0$ esiste $x\in A$ tale che:
\[x_0-\delta <x<x_0+\delta \qquad x \neq x_0\]
\item $+\infty$ è detto punto di accumulazione per $A$ se per ogni $R>0$ esiste $x\in A$ tale che $x>R$
\item $-\infty$ è detto punto di accumulazione per $A$ se per ogni $R>0$ esiste $x\in A$ tale che $x<-R$
\end{enumerate}

\section{Limite}
Data una funzione $f:A\rightarrow \R$, un punto di accumulazione $x_0\in\R\cup\{\pm\infty\}$ per $A$ ed $\ell\in\R\cup\{\pm\infty\}$, si scrive
\[\lim_{x\to x_0}{f(x)=\ell}\]
Si distinguono i casi:
\begin{enumerate}
\item $x_0\in \R$ e $\ell \in \R$: se per ogni $\epsilon > 0$ esiste $\delta > 0$ tale che:
\[\ell - \epsilon < f(x) < \ell + \epsilon \qquad \forall x\in A,x\neq x_0 \cap x_0 - \delta < x < x_0 + \delta\]

\item $x_0\in \R$ e $\ell = \pm \infty$: se per ogni $M > 0$ esiste $\delta > 0$ tale che:
\[\begin{cases}
f(x)>M & \text{se } \ell=+\infty \\
f(x)<-M & \text{se } \ell=-\infty
\end{cases}
\quad
\forall x\in A, x\neq x_0 \cap x_0 - \delta < x < x_0 + \delta
\]

\item $x_0 = \pm \infty$ e $\ell = \in \R$: se per ogni $\epsilon > 0$ esiste $R > 0$ tale che:
\[\ell-\epsilon<f(x)<\ell+\epsilon \quad \forall x\in A \cap
\begin{cases}
x>R & \text{se } x_0=+\infty \\
x<-R & \text{se } x_0=-\infty
\end{cases}\]

\item $x_0 = \pm \infty$ e $\ell = \in \R$: se per ogni $M > 0$ esiste $R > 0$ tale che:
\[\begin{cases}
f(x)>M & \text{se } \ell=+\infty \\
f(x)<-M & \text{se } \ell=-\infty 
\end{cases}
\quad \forall x\in A \cap
\begin{cases}
x>R & \text{se } x_0=+\infty \\
x<-R & \text{se } x_0=-\infty
\end{cases}
\]
\end{enumerate}

In tal caso, si dice che esiste finito il limite di $f$ per $x$ che tende a $x_0$ e vale $\ell$ oppure che $f(x)$ tende ad $\ell$ per $x$ che tende a $x_0$.

\begin{prop}[Continuità dei limiti]
Data $f:A\rightarrow \R$ ed $x_0\in A$ punto di accumulazione per $A$, $f$ è continua in $x_0$ se e solo se:
\[\lim_{x\to x_0}{f(x)}=f(x_0)\]
\end{prop}

\subsection{Limite destro e sinitro}
Data una funzione $f:A\rightarrow \R$ ed un punto $x_0 \in \R$ per $A$ tale che per ogni $\delta>0$
\[A\cap (-\delta,x_0)\neq \emptyset \qquad \text{e} \qquad A\cap (x_0,\delta)\neq \emptyset\]
si scrive:
\[\begin{cases}
\lim_{x\to x_{0^+}}{f(x)}=\ell_1 \in \R & \text{limite destro} \\
\lim_{x\to x_{0^-}}{f(x)}=\ell_2 \in \R & \text{limite sinistro}
\end{cases}\]
Se per ogni $\epsilon >0$ esiste $\delta > 0$ tale che:
\[\begin{cases}
\ell_1 - \epsilon < f(x) < \ell_1 + \epsilon \\
\ell_2 - \epsilon < f(x) < \ell_2 + \epsilon \\
\end{cases}
\qquad \forall x\in A \cap
\begin{cases}
x_0<x<x_0+\delta & \text{limite destro}\\
x_0-\delta<x<x_0 & \text{limite sinistro}\\
\end{cases}\]
Analoghe definizioni valgono se $\ell_{1,2}=\pm\infty$

\begin{prop}
Data una funzione $f:A\rightarrow \R$, un punto $x_0\in \R$ tale che per ogni $\delta>0$
\[A\cap(-\delta,x_0) \neq \emptyset \qquad \text{e} \qquad A\cap(x_0,\delta) \neq \emptyset)\]
Allora $x_0$ è un punto di accumulazione per $A$ e:
\[\text{esiste } \lim_{x\to x_0}f(x)=\ell \quad \Leftrightarrow \quad \text{esistono }
\begin{cases}
\lim_{x\to x_{0^+}}{f(x)}=\ell \\
\lim_{x\to x_{0^-}}{f(x)}=\ell
\end{cases}\]
\end{prop}

\begin{teo}[Algebra dei limiti]
Date due funzioni $f,g:A\rightarrow\R$ ed un punto $x_0\in\R$ di accumulazione per $A$, se esitono:
\[\lim_{x\to x_0}{f(x)}=\ell_1\in\R\cup\{\pm\infty\}\]
\[\lim_{x\to x_0}{f(x)}=\ell_2\in\R\cup\{\pm\infty\}\]
allora:
\begin{description}
\item[Somma:]
\[\lim_{x\to x_0}{f(x)+g(x)}= 
\begin{array}{ |*{4}{c|} }
\toprule
& \ell_2 \in \R & \ell_2 = +\infty & \ell_2 = -\infty \\
\midrule
\ell_1 \in \R & \ell_1 + \ell_2 & +\infty & -\infty \\
\midrule
\ell_1 = +\infty & +\infty & +\infty & \foi \\
\midrule
\ell_1 = -\infty & -\infty & \foi & -\infty \\
\bottomrule
\end{array}\]
Dove $\foi$= forma indeterminata $+\infty -\infty$

\item[Prodotto:]
\[\lim_{x\to x_0}{f(x)g(x)}= 
\begin{array}{ |*{6}{c|} }
\toprule
& \ell_2<0 & \ell_2=0 & \ell_2>0 & \ell_2=+\infty & \ell_2=-\infty\\
\midrule
\ell_1<0 & \ell_1 \ell_2 & 0 & \ell_1 \ell_2 & -\infty & +\infty \\
\midrule
\ell_1=0 & 0 & 0 & 0 & \foi & \foi \\
\midrule
\ell_1>0 & \ell_1 \ell_2 & 0 & \ell_1 \ell_2 & +\infty & -\infty \\
\midrule
\ell_1=+\infty & -\infty & \foi & +\infty & +\infty & -\infty \\
\midrule
\ell_1=-\infty & +\infty & \foi & -\infty & -\infty & +\infty\\
\bottomrule
\end{array}\]
Dove $\foi$= forma indeterminata $0 \infty$

\item[Rapporto:]
\[\lim_{x\to x_0}{\frac{f(x)}{g(x)}}=
\begin{array}{ |*{6}{c|} }
\toprule
& \ell_2<0 & \ell_2=0^{\pm} & \ell_2>0 & \ell_2=+\infty & \ell_2=-\infty\\
\midrule
\ell_1<0 & \frac{\ell_1}{\ell_2} & \mp\infty & \frac{\ell_1}{\ell_2} & 0 & 0 \\
\midrule
\ell_1=0 & 0 & \foi & 0 & 0 & 0 \\
\midrule
\ell_1>0 & \frac{\ell_1}{\ell_2} & \pm\infty & \frac{\ell_1}{\ell_2} & 0 & 0 \\
\midrule
\ell_1=+\infty & -\infty & \pm\infty & +\infty & \foi & \foi \\
\midrule
\ell_1=-\infty & +\infty & \mp\infty & -\infty & \foi & \foi \\
\bottomrule
\end{array}
\]
Dove $\foi$= forma indeterminata $\frac{0}{0}$ o $\frac{\infty}{\infty}$ e la notazione $l_2=0^\pm$ significa che:
\begin{enumerate}
\item esiste il limite
\[\lim_{x\to x_0}{g(x)}=0\]
\item esiste $\delta>0$ tale che per ogni $x\in A \cap (x_0-\delta,x_0+\delta), x\neq x_0$
\[\begin{cases}
g(x)>0 & \ell_2 = 0^+ \\
g(x)<0 & \ell_2 = 0^- \\
\end{cases}\]
Se $x_0 \in \R$ (analoga definizione se $x_0=\pm\infty$).
\end{enumerate}
\end{description}
\end{teo}

\begin{teo}[Limite funzione composta.]
Date due funzioni $f:A\rightarrow\R,y=f(x)$ e $g:B\rightarrow\R,z=g(y)$, tali che:
\begin{enumerate}
\item per ogni $x\in A$, allora $f(x)\in B,$
\item il punto $x_0$ è di accumulazione per $A$ ed esiste:
\[\lim_{x\to x_0}{f(x)}=y_0\in\R\cup\{+\infty\},\]
\item il punto $y_0$ è di accumulazione per $B$ ed esiste:
\[\lim_{y\to y_0}{g(x)}=\ell\in\R\cup\{+\infty\},\]
\end{enumerate}
Allora esiste:
\[\lim_{x\to x_0}{g(f(x))}=\ell\]
\paragraph{Nota:}
Le condizioni del teorema non sono sufficienti per assicurare l'esistenza del limite $\lim_{x\to x_0}{g(f(x))}=\ell$. Occorre aggiungere delle ipotesi tecniche, che però sono sempre verificate negli esercizi. Ad esempio, è sufficiente richiedere che una delle seguenti tre condizioni sia soddisfatta:
\begin{enumerate}
\item il punto $y_0$ non appartiene a $\dom{g}$;
\item la funzione $g$ è continua in $y_0$;
\item esiste $\delta>0$ tale che $f(x)\neq y_0$ per ogni $x\in A,x\neq x_0$ e $x_0-\delta\leq x\leq x_0+\delta$.
\end{enumerate}
\end{teo}

\section{Limiti agli estremi del dominio di definizione}
\subsection{Potenze}

\begin{align*}
\lim_{x\to+\infty}{x^b}&=+\infty         & b>0      &\\
\lim_{x\to+\infty}{x^b}&=0               & b<0      &\\
\lim_{x\to-\infty}{x^n}&=+\infty         & n\in \N, &\text{ n pari} \\
\lim_{x\to-\infty}{x^n}&=-\infty         & n\in \N, &\text{ n dispari} \\
\lim_{x\to-\infty}{x^{-n}}&=0            & n\in \N, &n\geq 1 \\
\lim_{x\to-\infty}{\sqrt[n]{n}}&=-\infty & n\in \N, &\text{ n dispari}
\end{align*}

\subsection{Esponenziali e logaritmi}
\begin{align*}
&\lim_{x\to+\infty}{\e^x}=+\infty & \qquad & \lim_{x\to+\infty}{\ln x}=+\infty \\
&\lim_{x\to-\infty}{\e^x}=0 & \qquad & \lim_{x\to 0}{\ln x}=-\infty \\
&\lim_{x\to+\infty}{a^x}=+\infty & \qquad & a>1 \\
&\lim_{x\to+\infty}{a^x}=0 & \qquad & 0<a<1 \\
&\lim_{x\to-\infty}{a^x}=0 & \qquad & a>1 \\
&\lim_{x\to-\infty}{a^x}=+\infty & \qquad & 0<a<1 \\
&\lim_{x\to+\infty}{\log_a x}=+\infty & \qquad & a>1 \\
&\lim_{x\to+\infty}{\log_a x}=-\infty & \qquad & 0<a<1 \\
&\lim_{x\to 0}{\log_a x}=-\infty & \qquad & a>1 \\
&\lim_{x\to 0}{\log_a x}=+\infty & \qquad & 0<a<1
\end{align*}

\subsection{Funzioni trigonometriche ed inverse}
\begin{align*}
&\lim_{x\to\frac{\pi}{2}^-}{\tan x}=+\infty \\
&\lim_{x\to\frac{\pi}{2}^+}{\tan x}=-\infty \\
&\lim_{x\to-\frac{\pi}{2}^-}{\tan x}=+\infty \\
&\lim_{x\to-\frac{\pi}{2}^+}{\tan x}=-\infty \\
&\nexists \lim_{x\to\pm\infty}{\sin x} \\
&\nexists \lim_{x\to\pm\infty}{\cos x} \\
&\nexists \lim_{x\to\pm\infty}{\tan x} \\
&\lim_{x\to+\infty}{\arctan x}=\frac{\pi}{2} \\
&\lim_{x\to-\infty}{\arctan x}=-\frac{\pi}{2}
\end{align*}

\subsection{Forme indeterminate del tipo 0/0}
\begin{align*}
&\lim_{x\to0}{\frac{\sin x}{x}}= 1                & \qquad & \lim_{x\to0}{\frac{1-\cos x}{x^2}}=\frac{1}{2} \\
&\lim_{x\to0}{\frac{x-\sin x}{x^3}}=\frac{1}{6}   & \qquad & \lim_{x\to0}{\frac{\arctan x}{x}}= 1 \\
&\lim_{x\to0}{\frac{\e^x-1}{x}}= 1                & \qquad & \lim_{x\to0}{\frac{a^x-1}{x}}= \ln a \quad a>0 \\
&\lim_{x\to0}{\frac{\ln(1+x)}{x}}= 1              & \qquad & \\
&\lim_{x\to0}{\frac{\log_a(1+x)}{x}}= \frac{1}{\ln a} \quad a>0,a\neq1 & \qquad & \\
&\lim_{x\to0}{\frac{(1+x)^b-1}{x}}=b \quad b\in\R & \qquad & 
\end{align*}

\subsection{Forme indeterminate del tipo infinito/infinito o 0infinito}
\begin{align*}
&\lim_{x\to+\infty}{\frac{a^x}{x^b}}=+\infty & \qquad & a>1,b>0 & \\
&\lim_{x\to+\infty}{\frac{x^b}{\log_a x}}=+\infty & \qquad & a>1,b>0 & \\
&\lim_{x\to-\infty}{|x|^b a^x}=0 & \qquad & a>1,b>0 & \\
&\lim_{x\to0}{x^b \log_a x}=0 & \qquad & a,b>0,a\neq 1 &
\end{align*}

\section{Intorno}
Dato $x_0\in\R\cup\{\pm\infty\}$, un insieme $I$ della forma:
\begin{equation*}
I=
\begin{cases}
(x_0-r,x_0+r) \qquad r>0 \quad se x_0\in\R \\
(\R,+\infty) \qquad R>0 \quad se x_0=+\infty \\
(-\infty,-\R) \qquad R>0 \quad se x_0=-\infty
\end{cases}
\end{equation*}
è detto intorno di $x_0$.

\begin{teo}[Teorema del confronto.]
Data una funzione $f:A\rightarrow\R$ ed un punto $x_0$ di accumulazione per $A$, se:
\begin{enumerate}
\item esistono due funzioni $g,h:A\rightarrow\R$ tali che:
\[g(x)\leq f(x)\leq h(x) \qquad \forall x\in A\cap I, x\neq x_0\]
dove $I$ è un opportuno intorno di $x_0$.
\item esistono i limiti:
\[\lim_{x\to x_0}{g(x)}=\ell \qquad \text{e} \qquad \lim_{x\to x_0}{h(x)}=\ell\]
dove $\ell \in \R \cup \{\pm\infty\}$.
\end{enumerate}
Allora esiste:
\[\lim_{x\to x_0}{f(x)}=\ell\]
\end{teo}

\section{Limiti di successioni}
Una successione è una funzione definita sui numeri naturali:
\[f:\N\rightarrow\R \qquad f(n)=a_n \qquad n\in\N,\]
denotata con $(a_n)_{n\in\N}$ oppure:
\[a_0,a_1,a_2,\dots,a_n\]
Poichè $\N$ non è superiormente limitato, $x_0=+\infty$ è un punto di accumulazione per $\N$ e, se esiste, si denota con:
\[\lim_{n\to+\infty}{a_n}=\ell\in\R\cup\{\pm\infty\}\]
Valgono tutti i teoremi visti per i limiti di funzioni.

\begin{teo}[Caratterizzazione per successioni.]
Data uan funzione $f:A\rightarrow\R,y=f(x)$, ed un punto $x_0$ di accumulazione per $A$ sono fatti equivalenti:
\begin{aenumerate}
\item esiste:
\[\lim_{x\to x_0}{f(x)}=\ell\in\R\cup\{\pm\infty\}\]
\item per ogni successione $(a_n)_{n\in\N}$ tale che:
\begin{align*}
&a_n\in A & \qquad & a_n\neq x_0 & \qquad & \lim_{n\to+\infty}{a_n}=x_0
\end{align*}
allora esiste:
\[\lim_{n\to+\infty}{f(a_n)}=\ell\in\R\cup\{\pm\infty\}\]
\end{aenumerate}
\end{teo}

\section{Estremo superiore, inferiore, massimo e minimo assoluto.}
Data una funzione $f:A\rightarrow\R$,
\begin{enumerate}
\item un elemento $y_0\in\R$ è detto un maggiorante di $\im f$ se:
\[f(x)\leq y_0 \qquad \forall x\in A\]
inoltre, se esiste un maggiorante, $f$ si dice superiormente limitata.

\item un elemento $M\in\R$ è detto estremo superiore di $f$ se:
\begin{equation*}\begin{cases}
f(x)\leq M \qquad \forall x\in A \\
\forall\epsilon>0 \exists x\in A | f(x)>M-\epsilon
\end{cases}\end{equation*}
e si scrive $M=\sup_{x\in A}{f(x)}$.
Se $f$ non è superiormente limitata, si pone:
\[\sup_{x\in A}{f(x)}=+\infty\]

\item $x_M \in A$ è detto punto di massimo assoluto se:
\[f(x)\leq f(x_M) \qquad \forall x\in A\]
e $f(x_M)=\max_{x\in A}{f(x)}$ è detto massimo assoluto di $f$.

\item un elemento $y_0 \in\R$ è detto un minorante di $A$ se:
\[f(x)\geq y_0 \qquad \forall x\in A\]
e, se esiste un minorante, $f$ si dice inferiormente limitata.

\item un elemento $x_m\in\R$ è detto punto di minimo assoluto di $f$ se:
\[f(x)\geq f(x_m) \qquad \forall x\in A\]
e $f(x_m)=\min_{x\in A}{f(x)}$ è detto minimo assoluto di $f$

\item un elemento $m\in\R$ è detto estremo inferiore se:
\begin{equation*}\begin{cases}
f(x)\geq m \qquad \forall x\in A \\
\forall \epsilon>0\exists x\in A | f(x)<m+\epsilon
\end{cases}\end{equation*}
e si scrive $m=\inf_{x\in A}{f(x)}$.
Se $f$ non è inferiormente limitata, si pone:
\[\inf_{x\in A}{f(x)}=-\infty\]

\item $f$ è detta limitata se è inferiormente e superiormente limitata, cioè se esistono $m,M\in\R$ tali che:
\[m\leq f(x) \leq M \qquad \forall x\in I\]
\end{enumerate}

\paragraph{Osservazione}
Data una funzione $f:A\rightarrow\R$,
\begin{aenumerate}
\item se $x_m\in A$ è un punto di minimo assoluto, allora:
\[\min_{x\in A}{f(x)}=\inf_{x\in A}{f(x)}=f(x_m)\]
\item se $x_M\in A$ è un punto di massimo assoluto, allora:
\[\max_{x\in A}{f(x)}=\sup_{x\in A}{f(x)}=f(x_M)\]
\item se $f$ è limitata, allora:
\[\im f \subseteq [\inf_{x\in A}{f(x)}, \sup_{x\in A}{f(x)}] \]
\end{aenumerate}