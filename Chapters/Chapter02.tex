% Chapter 2

\chapter{Funzioni elementari di variabile reale} % Chapter title

\label{ch:funzioni-elementari} % For referencing the chapter elsewhere, use \autoref{ch:introduction} 

%----------------------------------------------------------------------------------------

\section{Il concetto di funzione}
\textbf{Definizione:} una funzione $f: A \rightarrow \R$ dove $A \subseteq \R$ è una legge che assegna ad ogni $x\in A$ uno ed un solo valore $y=f(x) \in \R$

\textit{Nota:} in questo caso, i valori di $A$ sono chiamati variabile indipendente $(x)$, mentre  $\R$ è la variabile dipendente $y=f(x)$

\textit{Nota:} inoltre definiamo $A=dom \quad f$ come il dominio della funzione.

\textbf{Definizione:} Il grafico di $f$:
	\[f=\left\{ (x,y) \in \R^2 \middle| x \in A, y=f(x) \right\}\]

\textbf{Definizione:} L'immagine di $f$, Im f:
	\[f(A)=\left\{ f(x) \in \R \middle| x \in A \right\}\]

\section{Operazioni tra funzioni}
Date due funzioni $f:A\rightarrow \R \qquad g:B\rightarrow \R$
\begin{description}
	\item[Somma e differenza:] $(f+g)(x)=f(x)+g(x) \qquad dom(f+g)=A\cap B$
	\item[Prodotto:] $(fg)(x)=f(x)g(x) \quad dom(fg)=(A\cap B)$
	\item[Rapporto:] $(\frac{f}{g})(x)=f(x)g(x) \quad dom(\frac{f}{g})=\{x\in \R | x\in A, x\in B, g(x) \neq 0\}$
	\item[Reciproco:] $\frac{1}{f}(x)=\frac{1}{f(x)}=[f(x)]^{-1} \quad dom(\frac{1}{f})={x\in A | f(x) \neq 0}$
\end{description}

\subsection{Nomenclatura}
Data una funzione $f: A \rightarrow \R , \quad y=f(x)$
\begin{itemize}
	\item $f$ è detta \textbf{iniettiva} se $\forall y_0 \in \R , f(x)=y_0$ ha al più una soluzione.
	\item $f$ è detta \textbf{surgettiva} se $\forall y_0 \in \R , f(x)=y_0$ ha almeno una soluzione.
	\item $f$ è detta \textbf{bigettiva} se $\forall y_0 \in \R , f(x)=y_0$ ha una ed una sola soluzione, ovvero se la funzione è sia iniettiva che surgettiva.
\end{itemize}

\subsubsection{Osservazioni}
\begin{enumerate}
	\item $f$ è surgettiva se e solo se $IM f=\R$
	\item $f$ è iniettiva se e solo se $y_0 \in IM f, f(x)=y_0$ ha al più una soluzione.
\end{enumerate}

Data una funzione $f: A \rightarrow \R , \quad y=f(x)$ sono fatti equivalenti:
\begin{itemize}
	\item $f$ è iniettiva
	\item $\forall x_1, x_2 \in A \cap x_1 \neq x_2$ allora $f(x_1)\neq f(x_2)$
	\item dati $x_1,x_2 \in A | f(x_1)=f(x_2)$ allora $x_1=x_2$
\end{itemize}


\section{Funzioni pari e dispari}
Data una funzione $f: A \rightarrow \R , \quad y=f(x)$, $\forall x\in A \quad -x\in A$ f è detta:
\[f(-x)= \begin{cases}
	f(x) \qquad pari\\
	-f(x) \qquad dispari
\end{cases}
\]

\section{Funzioni monotone}
Data una funzione $f: A \rightarrow \R , \quad y=f(x)$
\begin{itemize}
	\item $\forall x_1, x_2 \in A \quad x_1<x_2$ f è detta:
	\[\begin{cases}
		f(x_1) \leq f(x_2) \qquad crescente\\
		f(x_1) \geq f(x_2) \qquad decrescente
	\end{cases}
	\]
	\item $\forall x_1, x_2 \in A \quad x_1<x_2$ f è detta:
	\[\begin{cases}
		f(x_1) < f(x_2) \qquad strettamente crescente\\
		f(x_1) > f(x_2) \qquad strettamente decrescente
	\end{cases}
	\]
\end{itemize}


\section{Traslazioni, dilatazioni e riflessioni}
Data una funzione $f: A \rightarrow \R , \quad y=f(x)$:
\begin{description}
	\item[Traslazioni:] $x_0 > 0, \quad y_0 \in \R$
		\[g(x)=f(x-x_0) \text{ Traslazione verso destra}\]
		\[g(x)=f(x+x_0) \text{ Traslazione verso sinistra}\]
		\[g(x)=f(x)+y_0 \text{ Traslazione verso l'alto}\]
		\[g(x)=f(x)-y_0 \text{ Traslazione verso il basso}\]
	\item[Dilatazioni:] $a>0$
		\[g(x)=f(\frac{x}{a}) \text{ Dilata su asse x}\]
		\[g(x)=a\times f(x) \text{ Dilata su asse y}\]
	\item[Riflessioni:]
		\[g(x)=f(-x) \text{ Riflette su asse }y\]
		\[g(x)=-f(x) \text{ Riflette su asse }x\]
		\[g(x)=-f(-x) \text{ Riflette rispetto l'origine}\]
\end{description}


\subsection{Osservazioni}
Se $f(x)$ è dispari e $0 \in \text{dom f}$
\[f(0)=f(-0)=-f(0)\Rightarrow f(0)=0\]
\newline
Se $n \in \N, n \geq 1$
\[f(x)=x^n= \underbrace{x \times \dots \times x}_{\textbf{n volte}}\]
\begin{itemize}
	\item se $n$ è pari, $f$ è pari
	\item se $n$ è dispari, $f$ è dispari
\end{itemize}

\section{Simmetrie, traslazioni, compressioni e dilatazioni di grafici.}
Data una funzione $f: A \rightarrow \R , \quad y=f(x)$:
\begin{description}
	\item[Traslazioni:] $x_0 > 0, \quad y_0 \in \R$
		\[g(x)=f(x-x_0) \text{ Traslazione verso destra}\]
		\[g(x)=f(x+x_0) \text{ Traslazione verso sinistra}\]
		\[g(x)=f(x)+y_0 \text{ Traslazione verso l'alto}\]
		\[g(x)=f(x)-y_0 \text{ Traslazione verso il basso}\]
	\item[Dilatazioni:] $a>0$
		\[g(x)=f(\frac{x}{a}) \text{ Dilata su asse x}\]
		\[g(x)=a\times f(x) \text{ Dilata su asse y}\]
	\item[Riflessioni:]
		\[g(x)=f(-x) \text{ Riflette su asse }y\]
		\[g(x)=-f(x) \text{ Riflette su asse }x\]
		\[g(x)=-f(-x) \text{ Riflette rispetto l'origine}\]
\end{description}

\section{Funzione composta}
Date due funzioni $f:A\rightarrow \R$ e $g:B\rightarrow \R$ la funzione:
\[g(y)=g(f(x))=(g\circ f)(x) \qquad x \in A\]
Con dominio:
\[\dom (g\circ f)=\{x\in \R | x\in A \cap f(x) \in B \}\]

\section{Funzione inversa e sue proprietà.}
Data una funzione iniettiva $f:A\rightarrow \R$
\[\forall y \in f=f(A), \exists ! x\in A | f(x)=y\]
Da cui si ricava che:
\[x=f^{-1}(y) \qquad f^{-1}: B\rightarrow \R \qquad B=Im f\]
\subsection{Costruire l'inverso di f}
\begin{enumerate}
	\item Determinare $Im f=B$ e $dom f^{-1}=B$
	\item $y \in B$ determiniamo $x \in A | f(x)=y$
	\item $x=f^{-1}(y)$
	\item $y=f^{-1}(x) \qquad x\rightleftarrows y$
\end{enumerate}
Il grafico di $y=f^{-1}(x)$ è simmetrico rispetto alla bisettrice $x=y$ della funzione $y=f(x)$
\subsubsection{Osservazioni}
\[f(f^{-1}(y)) = y \qquad \forall y \in dom^{f^{-1}}=Im f \]
\[f^{-1}(f(x)) = x \qquad \forall x \in dom f = Im f^{-1}\]
Inoltre $f$ è invertibile se e solo se è iniettiva o surgettiva, da cui: 
\[g^{-1}:Im f \rightarrow \R\]


\section{Polinomi}
\[f(x)=a_0+a_1 x + \dots + a_n x^n =\displaystyle\sum_{k=0}^{n} a_k x^k \]
\[a_0, a_1, \dots, a_n \in \R \text{ Coefficienti}\qquad a_n \neq 0 \text{ n è il grado del polinomio}\]

Per cui:
\[n=1 \qquad y=a_0+a_1 x \quad \text{Rette}\]
\[n=2 \qquad y=a_0+a_1 x + a_2 x^2 \quad \text{Parabole}\]
