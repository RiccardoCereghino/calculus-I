% Chapter X

\chapter{Funzioni continue} % Chapter title

\label{ch:continue} % For referencing the chapter elsewhere, use \autoref{ch:name}

%----------------------------------------------------------------------------------------

\section{Funzioni continue}

Data una funzione $f:A\rightarrow \R, y = f(x)$, ed un punto $x_0 \in A $, la funzione è detta continua in $x_0$ se per ogni $\epsilon > 0$ esiste $\delta > 0$ tale che:
\[f(x_0)-\epsilon < f(x) < f(x_0) + \epsilon \qquad \forall x \in A \cap x_0 - \delta < x < x_0 + \delta \]

La funzione è detta continua se è continua in $x_0$ per ogni $x_0 \in A$.

\begin{teo}[Continuità funzioni elementari]
Le funzioni potenza $x^a$, esponenziali $a^x$, logaritmo $\log_a x$, trigonometriche e trigonometriche inverse, sono continue.
\end{teo}

\begin{teo}[Algebra delle funzioni continue]
Date due funzioni $f,g: A \rightarrow \R, \quad y=f(x), \quad y=g(x)$, continue, allora:
\begin{enumerate}
\item la somma $f(x) + g(x)$ è una funzione continua;
\item il prodotto $f(x)g(x)$ è una funzione continua;
\item il rapporto $\frac{f(x)}{g(x)}$ è una funzione continua sul suo dominio ${x\in A | g(x) \neq 0 }$
\end{enumerate}
\end{teo}

\begin{teo}[Continuità funzione composta]
Date due funzioni continue $f:A\rightarrow \R$ e $g:B\rightarrow \R$ allora la funzione composta:
\[g\circ f:{x\in A |f(x) \in B} \rightarrow \R \qquad y=g(f(x))\]
è continua.
\end{teo}

\begin{teo}[Continuità funzioni inversa]
Data una funzione $f:I\rightarrow \R, \quad y=f(x)$, tale che:
\begin{enumerate}
\item $f$ è iniettiva;
\item $f$ è continua;
\item il dominio di $I$ è un intervallo;
\end{enumerate}
allora, posto $B = \im f$, la funzione inversa $f^{-1}:B\rightarrow \R$ è continua.
\end{teo}
