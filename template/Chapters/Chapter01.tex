% Chapter 1

\chapter{Notazione} % Chapter title

\label{ch:notazione} % For referencing the chapter elsewhere, use \autoref{ch:introduction} 

%----------------------------------------------------------------------------------------
Un richiamo alla notazione che verrà utilizzata nel documento.

\section{Insiemi}
\begin{tabular}{ll}
  $\emptyset$ & Insieme vuoto \\
  $\N$ & Naturali compreso lo $0$ \\
  $\Z$ & Relativi \\
  $\Q = \left\{ \frac{n}{m} \middle| n \in \Z , m \in \Z , m \neq 0 \right\} $ & Razionali \\
  $\R$ & Reali \\
\end{tabular}

\section{Logica}
\begin{tabular}{ll}
  $| $ & tale che \\
  $ \Rightarrow $ & implica \\
  $ \Leftrightarrow $ & se e solo se \\
  $ \forall $ & per ogni \\
  $ \exists $ & esiste \\
  $ \nexists $ & non esiste \\
  $ \in $ & appartiene \\
  $ \notin $ & non appartiene \\
\end{tabular}


\section{Operazioni tra insiemi}
\begin{description}
  \item[A sottoinsieme di B]
    \[A \subseteq  B\]
    \[\forall x \in A \Rightarrow x \in B\]
  \item[A sottoinsieme proprio]
    \[A \subsetneqq  B\]
    \[
      \begin{cases}
        \forall x \in A \Rightarrow x \in B \\
        \exists x \in B | x \notin A
        \end{cases}
      \]
    \item[Intersezione]
      \[A \cap B = \left\{ x \in X \middle| x \in A, x \in B\right\}\]
    \item[Unione]
      \[A \cup B = \left\{ x \in X \middle| x \in A or x \in B\right\}\]
    \item[Differenza insiemistica]
      \[A \diagdown B = \left\{ x \in X \middle| x \in A, x \notin B\right\}\]
    \item[Complementare]
      \[A^C = \left\{ x \in X \middle| x \notin A \right\}\]
\end{description}

\section{Prodotto cartesiano}
Assegnati due numeri reali $a, b, a < b$, si definiscono intervalli gli insiemi seguenti:
\[A\times B = \left\{ (x,y) \middle| x\in A, y\in B\right\}\]
Coppia ordinata: $(1,3) \neq (3,1)$


\subsection{Intervalli}
\begin{tabular}{ll}
intervallo limitato chiuso &$[a,b] = \left\{ x\in \R \middle| a \leq x \leq b \right\}$ \\
intervallo limitato aperto&$(a,b) = \left\{ x\in \R \middle| a < x < b \right\}$ \\
intervallo limitato aperto a destra&$[a,b) = \left\{ x\in \R \middle| a \leq x < b \right\}$ \\
intervallo limitato aperto a sinistra&$(a,b] = \left\{ x\in \R \middle| a < x \leq b \right\}$ \\

intervallo illimitato chiuso a sinistra&$[a,+\infty) = \left\{ x\in \R \middle| x \geq a \right\}$ \\
intervallo illimitato aperto a sinistra&$(a,+\infty) = \left\{ x\in \R \middle| x > a \right\}$ \\
intervallo illimitato chiuso a destra&$(-\infty,b] = \left\{ x\in \R \middle| x \leq b \right\}$ \\
intervallo illimitato aperto a destra&$(-\infty,b) = \left\{ x\in \R \middle| x < b \right\}$ \\
intervallo illimitato&$(-\infty,+\infty) = \R$ \\

\end{tabular}


\section{Proprietà delle operazioni tra numeri reali}
\[x,y,z \in \R\]
\begin{itemize}
  \item Associativa
    \[(x+y)+z=x+(y+z)=x+y+z \]
    \[(xy)z=x(yz)=xyz \]
  \item Commutativa
    \[x+y=y+x\]
    \[xy=yx\]
  \item Distributiva
    \[x(y+z)=xy+xz\]
  \item Esistenza elemento neutro
    \[x+0=x\]
    \[1x=x\]
  \item Esistenza dell'universo
    \begin{enumerate}
      \item \[\forall x \in \R \quad \exists !y=-x \in \R | x+(-x)=0\]
      \item \[\forall x \in \R \quad x \neq 0 \quad \exists !y=\frac{1}{x} \in \R | x\frac{1}{x}=1 \]
    \end{enumerate}
  \item Relazione d'ordine totale
    \[x,y,z \in \R\]
    \[
    \begin{cases}
      x<y \quad oppure\\
      x=y \quad oppure\\
      x>y
    \end{cases}
    \]
  \item Transitività
    \[x,y,z \in \R\]
    \[(x<y) \cap (y<z) \Rightarrow (x<z)\]
  \item Compatibilità con la somma
    \[x,y,z \in \R\]
    \[x<y \Rightarrow x+z<y+z\]
  \item Compatibilità con il prodotto
    \[x,y,z \in \R\]
    \[x<y \cap z>0 \Rightarrow xz<yz\]
    \[x<y \cap z<0 \Rightarrow xz>yz\]
\end{itemize}

\section{Geometria}
\subsection{Circonferenza}
Dato il centro di una circonferenza $C=(x_c,y_c)$
Si esprime l'equazione della circonferenza nella forma:
\[(x-x_c)^2+(y-y_c)^2=r^2\]

\[x^2+y^2+\alpha x+ \beta y +\gamma =r^2\]
Per cui se $O=(0,0)$
\[x^2+y^2=r^2\]


\paragraph{Forma canonica:}
\[\alpha = -2x_c \quad \beta = -2y_c \quad \gamma = x_c^2 + y_c^2 - r^2\]
\[x^2+y^2+\alpha x+ \beta y +\gamma =r^2\]

Da cui calcolare centro e raggio:
\[C=\left(-\frac{\alpha}{2}, -\frac{\beta}{2}\right); \quad r=\sqrt{\frac{\alpha^2}{4}+\frac{\beta^2}{4}-\gamma}\]

\subsection{Ellisse}
Equazione dell'ellisse (con centro nell'origine degli assi)
\[\frac{x^2}{a^2}+\frac{y^2}{b^2} \qquad a\neq 0 , b\neq 0\]
